\Question{Introduction}

In $\mod n$, we have $n$ equivalence classes $\{0, 1, \dots n-1\}$:

\begin{enumerate}
\item In regular arithmetic, $x^{-1}$ is defined so that $x x^{-1} = 1$. In this
case, $x^{-1} = \frac{1}{x}$. In modular arithmetic, we also define $x^{-1}$
such that $x x^{-1} = 1 \mod n$. e.g., $2^{-1} = 3 \mod 5$, since $2 \cdot 3 = 1 \mod 5$.
\item There are no negative numbers. In $\mod n$, add $n$ until your number $x$
satisfies $0 \leq x \leq n-1$.
\item There are no fractions, and division does not exist. Use multiplicative
inverse instead. e.g., $\frac{2}{5} = 2 \cdot 5^{-1} = 2 \cdot 3 \mod 7$.
\item Solve systems of equations as in regular arithmetic, keeping the above
rules in mind.
\end{enumerate}